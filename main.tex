\documentclass[twocolumn,amsthm]{autart}%

% Above is the standard document class.  If would like to use document
% class differently according to PDF or not, use the following:
% \RequirePackage{ifpdf}
% \ifpdf
% \documentclass[pdftex,. . . ]{. . . }
% \else
% \documentclass[. . . ]{. . . }
% \fi

%%%% PDF support
% Check if in PDF mode or not.  Usage: \\ifpdf ... \else ... \fi
\usepackage{ifpdf}


%%%% Graphic packages
% TODO: add more options/settings to this package
\usepackage{graphicx}
\usepackage{subcaption}
\graphicspath{{figs/}}

% If you don't want italic remark
% \newcounter{remark}
% \newenvironment{remark}{%
% \par\vspace{3pt}\noindent\refstepcounter{remark}\textbf{Remark~\theremark:}}%
% {\par\endtrivlist\unskip}

% The standard LaTeX font is Computer Modern. Here, we will load the New
% Century Schoolbook font instead. 
%\usepackage{newcent}
% Other popular options are (name followed by code)
% - Times:
% \usepackage{mathptmx}  % **
% - Palatino / Palladio: or \usepackage{tgpagella}
% \usepackage[sc]{mathpazo}  % **
% \linespread{1.05}         % Palladio needs more leading (space between lines)
% - Kp-fonts:
%\usepackage{kpfonts}  % *
% - GFS Didot:
%\usepackage{gfsdidot} % **
% - Utopia:
%\usepackage[utopia]{mathdesign} % *
% - Venturis:
%\usepackage{venturis} % **
% - Libertine:
%\usepackage{libertine} % **
% - Bitstream Vera / Dejavu:
% \usepackage{dejavu} % *

% Some other fonts that I know but haven't tested the quality: read texdoc psnfss2e
% - Helvetica
% \usepackage{helvet}
% - Avant Garde
% \usepackage{avant}
% - Bookman: Bookman for roman, Avant Garde for sans serif, and Courier
% \usepackage{bookman}
% - New Century: New Century for roman, Avant Garde for sans serif, and Courier
% \usepackage{newcent}
% - Charter
% \usepackage{charter}

% A little later, we will want to test something and behave 
% differently depending on the result. So the "ifthen" package is loaded
% to help out.
%\usepackage{ifthen}


% In TeX, fonts are "encoded", with each space in the font containing
% a different character. For most European users, the best setting is
% the "T1" (modern) system, which is set up next.
% \usepackage[T1]{fontenc}

% The fancyhdr package will allow us to alter the appearance of the
% header and footer of the page easily. This is much easier than trying
% to use the LaTeX kernel methods.
% \usepackage{fancyhdr}
% \pagestyle{fancy}
% \fancyhf{}
% \fancyhead[R]{Viet-Anh Le}
% \fancyhead[L]{Technical Notes}
% \fancyfoot[R]{Page \thepage}

\usepackage{parskip}
%\parskip=3mm   % Set the skip size between paragraphs

%%%% Math support (AMS math packages, symbols...)
% If you want to flush all equations to the left, or to set other
% options of amsmath, use the following line:
% \usepackage[fleqn]{amsmath}
\usepackage{mathtools,amsmath}
% Uncomment line below to allow page break inside multi-line equations
% \interdisplaylinepenalty=2500

% Additional math packages
\usepackage{amsfonts,amssymb}
\usepackage{bm} % for bold symbols like \bm{\alpha} or {\bm \alpha\beta}
\usepackage{mathrsfs}  % Formal math script font \mathscr
\usepackage{cases}  % Better than amsmath's cases environment
\usepackage{amsmath}
\usepackage{comment}

% indent the first paragraph after a section tag
\usepackage{indentfirst}
\setlength{\parindent}{5mm}

%%%% SI Unit support: should always use to typeset numeric data
% With option per-mode=symbol: kW/s, without it: kW s^{-1}
% Quick usage:
%   \num{1+-2i}
%   \num{.3e45}
%   \SI{number}{unit} or \si{unit}
% where unit is like:
%   kg.m.s^{-1} or \kilogram\metre\per\second
%   \si[per-mode=symbol]{\kilogram\metre\per\ampere\per\second} for
%        kg m / (A s)
\usepackage[per-mode=symbol]{siunitx}

% \usepackage{subfig}  % Sub-figures
% \usepackage{color}   % Support colors
\usepackage[svgnames, rgb]{xcolor}  % Support for color with standard names

% PGF/TikZ
\usepackage{tikz}
\usetikzlibrary{snakes,arrows,shapes,positioning}
% Some other TikZ libraries
% \usetikzlibrary{automata,calc,matrix}
% \usetikzlibrary{circuits.logic.US,circuits.ee.IEC}
% \usetikzlibrary{shapes.geometric,shapes.symbols,shapes.arrows}

% PGFplots to plot pretty graphs of data or functions
\usepackage{pgfplots}
\pgfplotsset{compat=1.15}
\usepackage{mathrsfs}
\usetikzlibrary{arrows}
% Define lengths for figure sizes, for PGFPlots figures exported from other tools
% \newlength{\figheight}
% \newlength{\figwidth}

% \usepackage{wrapfig}  $ Wrap text around figures

% \usepackage{vaucanson-g}  % Draw state machines


%%%% Special typesetting
% Algorithms
\usepackage[ruled]{algorithm}    % For creating floating algorithms, in the algorithms bundle

% The followings are packages for creating actual algorithms
\usepackage{algorithmic}  % Easy to use, Pascal-like, less flexible, in the algorithms bundle
%\usepackage{algorithm2e}  % More flexible, more like Pascal
%\usepackage{algpseudocode}  % In the algorithmicx package, flexible, more math-like

% For typesetting source code
%\usepackage{listings}


% Notes, comments
% \usepackage{todonotes}
% \usepackage{verbatim}   % better support for verbatims, comment environment

%%%% Customizations

% Customization of enumerate, itemize, description
%\usepackage{enumerate}  % Customize enumerate only
%\usepackage{enumitem}   % Customize enumerate, itemize, description

\usepackage{cite}  % Fix issues with \cite handling numbers
\usepackage{array}  % Better tabular and array
\usepackage{booktabs}           % More beautiful tables

%%%% Set-up hyperref for hyperlinks, etc.
%%%% This should always come last in the preamble.

\usepackage{amsthm}
\theoremstyle{definition}
\newtheorem{assumption}{Assumption}
\newtheorem{definition}{Definition}
% \newtheorem{remark}{Remark}
% \newtheorem{problem}{Problem}

\theoremstyle{plain}
\newtheorem{proposition}{Proposition}
\newtheorem{struct_result}{Structural Result}
\newtheorem{theorem}{Theorem}
\newtheorem{lemma}{Lemma}
\newtheorem{corollary}{Corollary}
\usepackage[extramath,probability,reviewmark]{mylatexdefs}

\DeclareMathOperator*{\stateinput}{
\begin{bmatrix}
\delta \mathbf{x}_{t} \\
\delta \mathbf{u}_{t}
\end{bmatrix}
}


\newcommand{\probP}{\text{I\kern-0.15em P}}
\newcommand{\GPModel}{\ensuremath{\GGG\PPP}}
\newcommand{\BLR}{\ensuremath{\BBB}}

\newcommand{\CAV}[1]{CAV\textendash\ensuremath{#1}\xspace}
\newcommand{\HDV}[1]{HDV\textendash\ensuremath{#1}\xspace}
\newcommand{\lane}[1]{lane\textendash\ensuremath{#1}\xspace}
\newcommand{\node}[1]{node\textendash\ensuremath{#1}\xspace}
\newcommand{\agent}[1]{agent\textendash\ensuremath{#1}\xspace}
\newcommand{\Agent}[1]{Agent\textendash\ensuremath{#1}\xspace}

\newcommand{\bb}[1]{\ensuremath{\mathbf{#1}}}
\newcommand{\bbsym}[1]{\ensuremath{\boldsymbol{#1}}}
\newcommand{\xtil}{\ensuremath{\Tilde{\bbsym{x}}}}


%%% Local Variables:
%%% mode: latex
%%% TeX-master: "main"
%%% End:


% The default \usepackage{hyperref} should work automatically.
% See hyperref for more options (visual, backlinks, etc.)
% Note that hyperref with pdftex seems fragile, especially with special symbols in section titles.
\ifpdf
  \usepackage[pdftex]{hyperref}
\else
  \usepackage[dvips]{hyperref}  % or dvipdfm
\fi

%%%%% BEGIN OF DOCUMENT

\begin{document}
\begin{frontmatter}
%\runtitle{Insert a suggested running title}  % Running title for regular
% papers but only if the title
% is over 5 words. Running title
% is not shown in output.
\title{Distributed Optimization Algorithm for Mixed Integer Quadratic Programming} % Title, preferably not more
% than 10 words. \thanksref{footnoteinfo}
% \thanks[footnoteinfo]{This research was supported in part by ARPAE's NEXTCAR program under the award number DE-AR0000796 and by the Delaware Energy Institute (DEI).}

\author[Paestum]{Viet-Anh Le}\ead{vietale@udel.edu},  % (ead) as shown
\author[Paestum]{Andreas A. Malikopoulos}\ead{andreas@udel.edu}
\address[Paestum]{Department of Mechanical Engineering, University of Delaware, 126 Spencer Lab, 130 Academy Street, Newark, DE, 19716, USA}  % Please supply
%\address[Rome]{Division of Systems Engineering and Center for Information and Systems Engineering, Boston University, 15 Saint Mary's Street,
%Brookline, MA, 02446, USA}             % full addresses
%\address[Baiae]{The White House, Baiae}        % here.
\begin{keyword}                           % Five to ten keywords,
.
\end{keyword}                             % keyword list or with the
% help of the Automatica
% keyword wizard


%%%%%%%%%%%%%%%%%%%%%%%%%%%%%%%%%%%%%%%%%%%%%%%%%%%%%%%%%%%%%%%%%%%%%%%%
\begin{abstract}

In this paper, we.

\end{abstract}
%%%%%%%%%%%%%%%%%%%%%%%%%%%%%%%%%%%%%%%%%%%%%%%%%%%%%%%%%%%%%%%%%%%%%%%%

\end{frontmatter}

% DO NOT ENTER CONTENTS DIRECTLY HERE
% PUT THEM IN SEPARATE TEX FILES, AND \input{part.tex} HERE

\section{Introduction}

\subsection{Literature Review on Mixed-Integer Optimization}

Review on mixed-integer optimization: \cite[Sec. 5.2]{ioan2021mixed}.

% \begin{itemize}
% \item Combinatorial Integral Approximation (CIA) \cite{burger2023gauss}.
% \item Branch-and-bound: \cite{stellato2018embedded}.
 
% \end{itemize}

% \Note{An idea is to combine CIA with distributed optimization algorithm like ADMM.}

% The CIA algorithm may involve three stages:
% \begin{enumerate}
% \item A relaxed version where the constraints for integer variables are replaced by box constraints is solved.
% \item The solutions for integer variables are
% obtained by minimizing some distance function.
% \item The integer controls are fixed and the optimization problem is solved again.
% \end{enumerate}

% In steps 1 and 3, the problems are QP so they can be solved efficiently with distributed optimization algorithms such as ADMM.

\cite{vujanic2016decomposition}, \cite{falsone2019decentralized}: A decomposition method for large-scale MILPs. 
A dualization approach with tightening of the coupling constraint was proposed to determine a feasible solution and provide a bound on the duality gap.
However, a central unit is still needed.
\cite{falsone2018distributed} extended to a distributed framework where agents exchange information only with their neighbors and no central unit is needed.

\cite{camisa2018primal}, \cite{camisa2021distributed}: 
Distributed algorithm based on primal decomposition.

% \cite{falsone2020tracking}: ADMM for distributed optimization with coupled constraints, no integer variables.

\cite{testa2019distributed}:

% \cite{liu2020efficient}:

The above papers considered MILPs, but the methods cannot be extended to MIQPs. The following papers consider MIQPs.

\cite{takapoui2020simple}: Use ADMM to solve MIQP, but not distributed.
% \Note{Even for a simple problem like integer linear programming, ADMM fails to converge or converge to the wrong optimal solution \cite{takapoui2020simple}.}

\cite{liu2022distributed}: MI optimization with coupled constraints, used relaxation and ADMM.
% This paper is pretty relevant to our initial algorithm.

\cite{liu2021distributed}: Considered MIQP with coupling constraints, and proposed a three-stage algorithm to solve the problem. However, the algorithm require an initial feasible guess of the solution.

\cite{sun2018distributed}: Decouple the large-scale problem into subproblems, each local update is a convex projection subproblem except one for the integrality constraint. Then a distributed projected subgradient algorithm is used.

\cite{yfantis2023hierarchical}, \cite{klostermeier2024numerical}: Dual decomposition-based distributed optimization for constraint-coupled mixed-integer programs.

% \cite{burger2023gauss}: CIA applied to MINLP, not distributed. 

\section{Problem Formulation}
\begin{definition}
Let $N$ be the total number of agents, $\VVV = \{1, \dots, N\}$ be the set of agents, and $\EEE \subset \VVV \times \VVV$ be the set of communication links.
For each \agent{i}, $i \in \VVV$, let $\NNN_i = \{j \in \VVV \;|\; (i,j) \in \EEE\}$ be the set of its neighbors.
\end{definition}

\begin{definition}
Let $\bbsym{x}_i$ be, $f_i (\bbsym{x}_i)$ be the local objective function,
where
\[
f_i (\bbsym{x}_i) = \bbsym{x}_i^\top \bbsym{Q}_i \bbsym{x}_i + \bbsym{q}_i^\top \bbsym{x}_i.
\]
We represent the coupling constraint between \agent{i} and \agent{j} in the following form
\[
\bbsym{A}^{j}_{i} \bbsym{x}_i + \bbsym{A}^{i}_{j} \bbsym{x}_j \le \bbsym{b}_{ij},
\]
\end{definition}

\section{Distributed Mixed-Integer Optimization with Proximal ADMM with Sequential Convex Programming}

First, we introduce an auxiliary variable $\bbsym{y}_{i}^{j}$ for \agent{i} involved in coupling constraint with \agent{j} and formulate the following equivalent problem,
\begin{equation}
\begin{split}
\underset{\bbsym{x}_i \in \XXX_i}{\min} & \quad \sum_{i=1}^N \; f_i (\bbsym{x}_i), \\
\text{subject to} & \quad \bbsym{A}^{j}_{i} \bbsym{x}_i \le \bbsym{y}_{i}^{j}, \, \forall j \in \NNN_i, \\
& \quad \bbsym{y}_{i}^{j} + \bbsym{y}_{j}^{i} = \bbsym{b}_{ij},\, \forall (i,j) \in \EEE.
\end{split}
\end{equation}
where.
We represent the local constraints in the following compact form, 
\begin{equation}
\bbsym{A}_{i} \bbsym{x}_i \le \bbsym{y}_{i},
\end{equation}
where $\bbsym{A}_{i}^\top = [(\bbsym{A}_{i}^{j})^\top]_{j \in \NNN_i}$, $\bbsym{y}_{i}^\top = [(\bbsym{y}_{i}^{j})^\top]_{j \in \NNN_i}$

To solve the MIQP problem, we combine proximal ADMM \cite{yang2022proximal} with sequential convexification of the integrality constraints as presented follows.
At each iteration $t$, we keep a mixed-integer-valued vector $\tilde{\bbsym{x}}_i^{(t)} \in \XXX_i$ and an real-valued solution of the relaxed problem $\bar{\bbsym{x}}_i^{(t)} \in \bar{\XXX}_i$ where $\bar{\XXX}_i$ is formed from $\XXX_i$ by relaxing the integrality constraints.
We define $\Delta (\bar{\bbsym{x}}_i, \tilde{\bbsym{x}}_i)$ as the distance function between the real-valued vector $\bar{\bbsym{x}}_i$ and and the mixed-integer-valued vector $\tilde{\bbsym{x}}_i$.
% The full details of the distance function are skipped now but can be found in \cite{bertacco2007feasibility}. 

The augmented Lagrangian for the problem is formulated as follows
\begin{equation}
\begin{multlined}
\LLL ( \{ \bbsym{x}_i, \bbsym{y}_i, \bbsym{\lambda}_{i} \}_{i\in \VVV}) = \sum_{i \in \VVV} \Big[ f_i (\bbsym{x}_i) + \II (\bbsym{A}_{i} \bbsym{x}_i \le \bbsym{y}_{i}) \\
+ \sum_{j \in \NNN_i} (\bbsym{\lambda}_{i}^j)^\top (\bbsym{y}_{i}^{j} + \bbsym{y}_{j}^{i} - \bbsym{b}_{ij}) \Big] 
+ \frac{\rho}{2} \sum_{(i,j) \in \EEE} \norm{\bbsym{y}_{i}^{j} + \bbsym{y}_{j}^{i} - \bbsym{b}_{ij}}_2^2
\end{multlined}
\end{equation}
where $\bbsym{\lambda}_{i}^j$ are the dual variables (Lagrangian multipliers), $\bbsym{\lambda}_{i}^\top = [(\bbsym{\lambda}_{i}^j)^\top]_{j \in \NNN_i}$ and $\rho$ is a positive constant.

The algorithm consists of the following steps
\begin{itemize}
\item \Agent{i} solve the local problem \eqref{eq:local_update}.
% \begin{figure*}
\begin{equation}
\label{eq:local_update}
\begin{split}
\bar{\bbsym{x}}_i^{(t+1)}, {\bbsym{y}}_i^{(t+1)} =\; & \underset{\bbsym{x}_i \in \bar{\XXX}_i, \bbsym{y}_i}{\argmin}  \; \LLL_i (\bbsym{x}_i, \bbsym{y}_i, \bbsym{\lambda}_{i}^{(t)}) \\
& + \beta^{(t)} \Big( \norm{\bbsym{x}_i - \tilde{\bbsym{x}}_i^{(t)}}_2^2 + \norm{\bbsym{y}_i - \bbsym{y}_i^{(t)}}_2^2 \Big) , \\
& \text{subject to} \; \bbsym{A}_{i} \bbsym{x}_i \le \bbsym{y}_{i},
\end{split}
\end{equation}
where $\beta^{(t)} \in \RRplus$ is a penalty weight and 
\begin{equation}
\LLL_i (\bbsym{x}_i, \bbsym{y}_i, \bbsym{\lambda}_{i}) =    
\end{equation}
% \end{figure*}

\item \Agent{i} transmits ${\bbsym{y}}_i^{j,(t+1)}$ to \agent{j}.
\item Update the dual variables $\bbsym{\lambda}_{i}$ by 
\begin{equation}
\bbsym{\lambda}_{i}^{j,(t+1)} = \bbsym{\lambda}_{i}^{j,(t)} + \gamma \rho \big( \bbsym{y}_{i}^{j,(t+1)} + \bbsym{y}_{j}^{i,(t+1)} - \bbsym{b}_{ij} \big).
\end{equation}
where $\alpha^{(t)} \in \RRplus$ is the step-size.
\item \Agent{i} transmits $\bbsym{\lambda}_i^{j,(t+1)}$ to \agent{j}.
\item Compute $\tilde{\bbsym{x}}_i^{(t+1)}$ from $\bbsym{x}_i^{(t+1)}$ by rounding operator (transform a real-valued solution into an integer one).
\end{itemize}

\Note{If $\bbsym{\lambda}_i^{j,(t+1)} = \bbsym{\lambda}_j^{i,(t+1)}$, then we do not need the commucation for the dual variables.}

\Note{In this algorithm, we only need to solve QP problem, do not need to solve any mixed-integer optimization problem.}

\textbf{Convergence Analysis}

To prove convergence, we need: (1) identifying a so-called sufficiently decreasing Lyapunov function (usually the augmented Lagrangian); and
(2) establishing the lower boundness property of the Lyapunov
function \cite{yang2022proximal}.

Dinh \etal \cite{dinh2013dual} proposed sequential convex programming (SCP) scheme based on penalty function approach to handle nonconvexity.
\cite{yang2022proximal} shows the convergence of proximal ADMM for nonconvex and nonsmooth optimization.

Some papers about feasibility pump in mixed-integer optimization: \cite{boland2012new} shows that feasibility pump might be equivalent to the proximal point algorithm.
\cite{geissler2017penalty} shows that feasibility pumps can be seen as alternating direction methods.

\cite[Section V]{deng2017parallel} discussed the connection between the Proximal Jacobian ADMM Algorithm and the proximal point algorithm.

\Note{In what follows, I just summarize the proof in \cite{deng2017parallel} for original Jacobi-Proximal ADMM.}

\begin{assumption}
There exists a saddle point $(\bbsym{x}^{*}, \bbsym{\lambda}^{*})$
\end{assumption}

To simplify the notation, we let
\begin{equation}
\bbsym{P} = 
\begin{pmatrix}
\bbsym{P}_{1} + \rho \bbsym{A}_i^\top \bbsym{A}_i \\
& \ddots \\
& & \bbsym{P}_N + \rho \bbsym{A}_N^\top \bbsym{A}_N
\end{pmatrix}
\end{equation}

% Define $\Delta \bbsym{\lambda}^{(t)} = (\bbsym{\lambda}^{(t)}-\bbsym{\lambda}^{*})$
% and $\Delta \bbsym{x}^{(t)} = (\bbsym{x}^{(t)}-\bbsym{x}^{*})$,
We consider the following Lyapunov function
\begin{equation}
\begin{multlined}
\Phi^{(t)} = \norm{\bbsym{x}^{(t)}-\bbsym{x}^{*}}^2_{\bbsym{P}}
+ \frac{1}{\gamma \rho} \norm{\bbsym{\lambda}^{(t)}-\bbsym{\lambda}^{*}}_2^2 \\
+ \eta \norm{\xtil^{(t)} - \bbsym{x}^{(t)}}_2^2
\end{multlined}
\end{equation}
where $\eta > 0$.
The overall idea for the convergence proof is to prove the above Lyapunov function is not increasing over the iterations.

\begin{lemma}
For $t \ge 1$, we have
\begin{equation}
\begin{multlined}
\Phi^{(t)} - \Phi^{(t+1)}  \\
\ge \norm{\xtil^{(t)}-\bbsym{x}^{(t+1)}}_{\bbsym{P}-\eta \bbsym{I}}^2
+ \frac{2-\gamma}{\rho \gamma^2} \norm{\bbsym{\lambda}^{(t)}-\bbsym{\lambda}^{(t+1)}}_2^2 \\
+ \frac{2}{\gamma} (\bbsym{\lambda}^{(t)}-\bbsym{\lambda}^{(t+1)})^\top \bbsym{A} (\xtil-\bbsym{x}^{(t+1)})
+ \eta \norm{\xtil^{(k)}-\bbsym{x}^{(k)}}_2^2
\end{multlined}
\end{equation}
\end{lemma}

\begin{proof}
Sketch:

From step 1 of the algorithm:

Since $\xtil_i^{(t+1)} = \underset{\bbsym{x}_i \in \XXX_i}{\argmin} \;\norm{\bbsym{x}_i - \bbsym{x}_i^{(t+1)}}_2^2$ (step 2 of the algorithm), we have
\begin{equation}
\norm{\xtil^{(t+1)} - \bbsym{x}^{(t+1)}}_2^2 
\le \norm{\xtil^{(t)} - \bbsym{x}^{(t+1)}}_2^2 
\end{equation}
\end{proof}
% \begin{theorem}
% \end{theorem}

\section{Numerical Studies}

\begin{comment}
\section{Optimization Algorithm}

We need to develop an optimization algorithm to efficiently solve not in centralized manner.
Two approaches have been found in the literature:

\begin{itemize}
\item Distributed optimization: \cite{huang2023decentralized}.
\item Sequential computation with priority assignment: \cite{katriniok2022fully}, \cite{scheffe2023reducing}.
\end{itemize}

Let assume that the TLC units keep the trajectory prediction of HDVs on each lane.

\section{Prediction Model for HDVs}

\Note{Will start doing simulation with constant acceleration model first.}

% If we consider the traffic light at time $t$ is known

\begin{itemize}
\item The human driving behavior may follow different modes (\eg car-following, constant velocity, or constant acceleration), and the human control policy in each mode is parameterized by parameters (\eg car-following model's parameters) that describe different human driving styles.

% \item Since there is no lateral conflict involving HDVs, we can use a car-following model to predict the future trajectories of HDVs.

% \item Can use interacting multi-modal Kalman filter for predicting HDV's trajectory with three modes: car-following, constant velocity, or stop by the traffic light.

\item Note that even if the original physical system is a linear system, the dynamics of the new system with parameters as augmented states can be nonlinear.

\item Some references: \cite{lefkopoulos2020interaction}, \cite{hu2022active}, \cite{arcari2020dual}.
\end{itemize}

Let $\MMM = \{ M_1, \dots, M_N \}$ be the set of operating modes, $M \in \MMM$ be the mode with associated uncertain parameters in a vector $\boldsymbol{\theta}^M$.
The dynamics of the vehicles under each mode can be given by
\begin{equation}
M \;:\; \bbsym{x}(k+1) = \bbsym{f}^M (\bbsym{x}(k), \bbsym{z}(k); \bbsym{\theta}^M) + \bbsym{w}^M (k),
\end{equation}
where $\bbsym{z}(k)$ is the vector of context variables, \eg states of preceding vehicles or states of the traffic light.

% \begin{figure}
% \centering
% \includegraphics[width=0.45\textwidth]{figs/singlelane.png}
% % \caption{Caption}
% % \label{fig:enter-label}
% \end{figure}

We need to identify both the operating mode $M$ and the parameters $\boldsymbol{\theta}^M$, \ie compute $p \big( M, \bbsym{\theta}^M \,|\, \III(k) \big)$ where $\III(k)$ is the vector of available information at time $k$, \ie $\III(k+1)^\top = [\III(k)^\top, \bbsym{x} (k+1)^\top, \bbsym{z} (k+1)^\top]$. 
The posterior joint distribution of mode $M$ and $\bbsym{\theta}^M$ is given by
\begin{equation}
p \big( M, \bbsym{\theta}^M \,|\, \III(k+1) \big)
= p \big( \bbsym{\theta}^M \,|\, \III(k+1), M \big) 
p \big( M \,|\, \III(k+1) \big).
\end{equation}

If the system states are fully known, and the dynamics can be represented in the following form:
\begin{equation*}
M \;:\; \bbsym{x}(k+1) = \bbsym{f}_0^M (\bbsym{x}(k), \bbsym{z}(k)) + \bbsym{\Phi}^M (\bbsym{x}(k), \bbsym{z}(k))\,  \bbsym{\theta}^M + \bbsym{w}^M (k),
\end{equation*}
then we have a closed-form solution from \cite{arcari2020dual}.

For our problem: (1) car-following, (2) constant velocity, or (3) constant acceleration.

\begin{itemize}
\item Mode 1: Using CTH-RV car-following model:
\begin{equation}
u_i(k) = \alpha (\Delta p_i(k) - \rho v_i(k)) + \beta \Delta v_t(k)
\end{equation}

Thus
\begin{equation}
\begin{split}
p_i(k+1) &= p_i(k) + \tau v_i(k) + \frac{1}{2} \tau^2 u_i (k) \\
v_i(k+1) &= v_i(k) + \tau u_i(k) \\
u_i(k) & = \theta_1^{M_1} (p_{i-1}(k) - p_{i}(k) -d_{\min}) + \theta_2^{M_1} v_i(k) + \theta_3^{M_1} (v_{i-1}(k) - v_i(k))
\end{split}
\end{equation}
where $d_{\min}$ is a constant minimum inter-vehicle distance.

Re-write in matrix form:
\[\bbsym{x}_i (k) = [p_i (k), v_i (k)]^\top,\]
\[\bbsym{z}_i (k) = \bbsym{x}_{i-1} (k),\]
\[\bbsym{f}_0^{M_1} = [p_i (k) + \tau v_i(k), v_i (k)]^\top\]
\[\bbsym{\Phi}^{M_1} = \begin{bmatrix}
\frac{1}{2} \tau^2 (p_{i-1}(k) - p_{i}(k) -d_{\min}) & \frac{1}{2} \tau^2 v_i(k) & \frac{1}{2} \tau^2 (v_{i-1}(k) - v_i(k)) \\ 
\tau (p_{i-1}(k) - p_{i}(k) -d_{\min}) & \tau v_i(k) & \tau (v_{i-1}(k) - v_i(k))
\end{bmatrix}\]

\item Constant velocity:
\begin{equation}
\begin{split}
p_i(k+1) &= p_i(k) + \tau v_i(k) + \frac{1}{2} \tau^2 u_i (k) \\
v_i(k+1) &= v_i(k) + \tau u_i(k) \\
u_i(k) &= k_v (v^{\text{ref}} - v_i(k)) = \theta_1^{M_2} + \theta_2^{M_2} v_i(k)
\end{split}
\end{equation}

Re-write in matrix form:
\[\bbsym{\Phi}^{M_2} = \begin{bmatrix}
\frac{1}{2} \tau^2 & \frac{1}{2} \tau^2 v_i(k) \\ 
\tau & \tau v_i(k) 
\end{bmatrix}\]

\item Constant acceleration:
\begin{equation}
\begin{split}
p_i(k+1) &= p_i(k) + \tau v_i(k) + \frac{1}{2} \tau^2 u_i (k) \\
v_i(k+1) &= v_i(k) + \tau u_i(k) \\
u_i(k) &= a^{\text{ref}} = \theta_1^{M_3}
\end{split}
\end{equation}

Re-write in matrix form:
\[\bbsym{\Phi}^{M_3} = \begin{bmatrix}
\frac{1}{2} \tau^2 \\ 
\tau 
\end{bmatrix}\]

\end{itemize}

\Note{Actually constant velocity can be merged with constant acceleration model.}

Bayesian estimation:
\begin{subequations} 
\label{eq:Bayesian_update}
\begin{align}
p(\bbsym{\theta}^{M}\,|\, \III(k+1), M) &= \frac{p(\bbsym{x}(k+1) \,|\, \bbsym{z}(k), \III_k, M, \bbsym{\theta}^{M}) p(\bbsym{\theta}^{M}\,|\,  \III(k), M)}{p(\bbsym{x}(k+1) \,|\, \bbsym{z}(k), \III(k), M)}, \label{eq:Bayesian_update_param}
\\
p(M \,|\, \III(k+1)) & = \frac{p(\bbsym{x}(k+1) \,|\, \bbsym{z}(k), \III(k), M) p(M \,|\, \III(k))}{p(\bbsym{x}(k+1) \, | \, \bbsym{z}(k) ,\III(k))}.
\label{eq:Bayesian_update_mode}
\end{align}
\end{subequations}

\textbf{For parameter estimation}

Assume that the distribution of parameter $\bbsym{\theta}^M$ at time step $k$ is given by a normal distribution
\begin{equation*}
p(\bbsym{\theta}^M \,|\, \III(k) , M) =  \NNN \big( \mu_{\bbsym{\theta}}^M (k), \Sigma_{\bbsym{\theta}}^M (k) ).
\end{equation*}
The likelihood of the state at time step $k+1$ can be computed as
\begin{equation*}
p(\bbsym{x} (k+1) \,|\, \bbsym{z} (k), \III (k), M, \bbsym{\theta}^M) 
= \NNN \big( \bbsym{\Phi}^M( \bbsym{x} (k),\bbsym{z} (k) ) \bbsym{\theta}^M, \ \Sigma^M_w \big).
\end{equation*}

The update equations for parameter mean and variance:
\begin{equation}
\begin{split}
& \left[ \Sigma_{\theta}^M (k+1) \right]^{-1} = \left[ \Sigma_{\theta}^M (k) \right]^{-1} + \Phi^M(\bbsym{x}(k), \bbsym{z}(k))^T \ \left[ \Sigma_w^{M} \right]^{-1} \ \Phi^M(\bbsym{x}(k), \bbsym{z}(k)), \\
& \mu_{\theta}^M (k+1) = \Sigma_{\theta}^M (k+1) \left( \left[ \Sigma_{\theta}^M (k) \right]^{-1} \mu_{\theta}^M (k) + \Phi^M(\bbsym{x}(k),\bbsym{z}(k))^T \ \left[ \Sigma_w^{M}\right]^{-1} \ \bbsym{x}(k+1) \right).
\end{split}
\end{equation}

That is equivalent to a Kalman filter for each mode.


\textbf{For mode estimation}

\begin{multline}
p(x(k+1) \,|\, z(k), \III(k), M) = \NNN( \Phi^M(x(k),z(k)) \mu_{\theta}^M(k), \\ \Sigma^M_w \ + \ \Phi^M(x(k),z(k)) \ \Sigma^M_{\theta}(k) \Phi^M(x(k),z(k))^T ),    
\end{multline}

Note that the denominator in \eqref{eq:Bayesian_update_mode} is computed by 
\begin{equation}
p(\bbsym{x}(k+1) \, | \, \bbsym{z}(k) ,\III(k))
= \sum_{n=1}^{N} p(x(k+1) \,|\, z(k), \III(k), M_n) p( M_n \,|\, \III(k)).
\end{equation}

Finally, we choose the mode with maximum probability $p \big( M \,|\, \III(k+1) \big)$.


\begin{equation}
p(M \,|\, \III(k+1)) = \frac{\NNN( \Phi^M(x(k),u(k)) \mu_{\gamma,k}^M , \Sigma^M_w + \Phi^M(x(k),u(k))  \Sigma^M_{\gamma,k} \Phi^M(x(k),u(k))^T )  p(M|\III(k))}{\sum_{m=1}^{n_m}{ \NNN( \Phi^{M^m}(x(k),u(k)) \mu_{\gamma,k}^{M^m} , \Sigma^{M^m}_w + \Phi^{M^m}(x(k),u(k)) \  \Sigma^{M^m}_{\gamma,k} \Phi^{M^m}(x(k),u(k))^T ) \ p(M^{m}|\III(k)) } },
\end{equation}

\Note{If the states are not fully observable. Should read: \cite{heirung2019model}.}

\Note{Even if the system model linear, the model for augmented states is nonlinear. IMM-KF \cite{lefkopoulos2020interaction} only work if the model with augmented states is linear.}
\end{comment}






%%%%  REFERENCES

\bibliographystyle{IEEEtran}
\bibliography{IEEEabrv,references}

\end{document}

%%% Local Variables: 
%%% mode: latex
%%% TeX-master: t
%%% End: 