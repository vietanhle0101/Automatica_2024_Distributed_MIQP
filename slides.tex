\documentclass[9pt]{beamer}
\usepackage[utf8]{inputenc}
\usepackage{comment}
%package for algorithm/pseudo code
\usepackage{algorithm}
\usepackage{algpseudocode}
\usepackage{mathtools}

\usetheme{Madrid}
\usecolortheme{default}

\usepackage{xspace}
\usepackage[style=authortitle ,backend=biber]{biblatex}
\addbibresource{references.bib}
% \addbibresource{references/IDS_publication.bib}

\usepackage{todonotes}

% Algorithms
% \usepackage[ruled]{algorithm}
% \usepackage[noend]{algpseudocode}

\usepackage[extramath,probability]{mylatexdefs}

\newcommand{\probP}{\text{I\kern-0.15em P}}
\newcommand{\GPModel}{\ensuremath{\GGG\PPP}}
\newcommand{\BLR}{\ensuremath{\BBB}}

\newcommand{\CAV}[1]{CAV\textendash\ensuremath{#1}\xspace}
\newcommand{\HDV}[1]{HDV\textendash\ensuremath{#1}\xspace}
\newcommand{\lane}[1]{lane\textendash\ensuremath{#1}\xspace}
\newcommand{\node}[1]{node\textendash\ensuremath{#1}\xspace}
\newcommand{\agent}[1]{agent\textendash\ensuremath{#1}\xspace}
\newcommand{\Agent}[1]{Agent\textendash\ensuremath{#1}\xspace}

\newcommand{\bb}[1]{\ensuremath{\mathbf{#1}}}
\newcommand{\bbsym}[1]{\ensuremath{\boldsymbol{#1}}}
\newcommand{\xtil}{\ensuremath{\Tilde{\bbsym{x}}}}


%%% Local Variables:
%%% mode: latex
%%% TeX-master: "main"
%%% End:


% All user-defined commands should be in defs.tex

%------------------------------------------------------------
%This block of code defines the information to appear in the
%Title page
\title[Research] %optional
{Technical Notes for Research Project} 

\subtitle{Summary}

\author[Viet-Anh Le] % (optional)
{Viet-Anh Le\inst{1,2}}

\institute[UD] % (optional)
{
\inst{1}%
University of Delaware
\\
\inst{2}%
Cornell University
}

%\date[VLC 2021] % (optional)
%{Very Large Conference, April 2021}

%\logo{\includegraphics[height=1cm]{overleaf-logo}}

%End of title page configuration block
%------------------------------------------------------------


%------------------------------------------------------------
%The next block of commands puts the table of contents at the
%beginning of each section and highlights the current section:

% \AtBeginSection[]
% {
%   \begin{frame}
%     \frametitle{Table of Contents}
%     \tableofcontents[currentsection]
%   \end{frame}
% }
%-----------------------------------------------------------
\begin{document}

%The next statement creates the title page.
\frame{\titlepage}


%---------------------------------------------------------
%This block of code is for the table of contents after
%the title page
% \begin{frame}
% \frametitle{Table of Contents}
% \tableofcontents
% \end{frame}
%---------------------------------------------------------


\begin{frame}{Reference 1}
\cite{suriyarachchi2023optimization}

\begin{columns}[onlytextwidth]
\column{.49\textwidth}
\begin{figure}
\includegraphics[height=3.4cm]{figs/ref1-1.png}
\end{figure}
\column{.49\textwidth}
\begin{figure}
\includegraphics[height=3.4cm]{figs/ref1-2.png}
\end{figure}
\end{columns}
\end{frame}

\begin{frame}{Reference 1}
\cite{suriyarachchi2023optimization}
\begin{itemize}
\item Joint optimization problem to find traffic light states and CAV trajectories.
\item Objective function: maximizes the traveled distance of all vehicles over a control horizon.
\item Collision avoidance: rear-end constraints.
\item HDVs are modeled by 3 travel modes: stop at the traffic light, follow a preceding vehicle, or travel at a constant speed.
\end{itemize}
\end{frame}

\begin{frame}{Reference 2}
\cite{ghosh2022traffic}
\begin{figure}
\includegraphics[height=5cm]{figs/ref2-1.png}
\end{figure}
\end{frame}

\begin{frame}{Reference 2}
\cite{ghosh2022traffic}

\begin{itemize}
\item The traffic controller determines the order at which the vehicles would enter the intersection.
\item Objective function: linear combination of (1) energy cost, (2) travel time, (3) maximization of the velocity at entry of the intersection, and (4) minimization of the overall time for all the vehicles to cross the intersection.
\item Safety constraints: vehicles from colliding lanes can not be present at the intersection at the same time.

\item HDVs are modeled by IDM if they follow other vehicles and constant accelerations if they face the traffic light transition. 

\end{itemize}

\end{frame}



\begin{frame}{Reference 3}
\cite{kamal2019development}
\begin{figure}
\includegraphics[height=5cm]{figs/ref3-1.png}
\end{figure}
\end{frame}


\begin{frame}{Reference 3}

\cite{kamal2019development}
\begin{itemize}
\item Objective function: minimizes the (approximate) total crossing times of all vehicles
\item Given the traffic signal, CAVs and HDVs are controlled by the IDM car-following model.
\item Not clear how to identify IDM parameters for HDVs.
\end{itemize}
\end{frame}


\begin{frame}{Reference 4}
\cite{guo2023cotv}
\begin{figure}
\includegraphics[height=3.4cm]{figs/ref4-1.png}
\end{figure}
\end{frame}

\begin{frame}{Reference 4}
\cite{guo2023cotv}
\begin{itemize}
\item The reward of traffic signal controller is the penalty of intersection pressure (defined as the difference between the number of vehicles on the incoming roads and the number of vehicles on the outgoing roads).
\item The reward of CAVs include the deviation of
average speed from the maximum speed limit, plus the Euclidean norm of acceleration.
\item Not clear how to avoid collisions.
\item HDVs are simulated by IDM during training.
\end{itemize}
\end{frame}

\begin{frame}{Reference 5}
\cite{du2021coupled}
\begin{figure}
\includegraphics[height=4cm]{figs/ref5-1.png}
\end{figure}
\end{frame}

\begin{frame}{Reference 5}
\cite{du2021coupled}

\begin{itemize}
\item Signal timing optimization to find a set of green time. 
\item Objective: a weighted sum of two terms (1) the delay time for all delayed vehicle (= average time headway $\times$ number of vehicles in the lane) and (2) the difference among all phases to ensure fairness.
\item CAV eco-driving trajectory is generated by planned arrival time and the trigonometric function algorithm.
\item HDVs' trajectories are predicted by IDM, same parameters for all HDVs.
\end{itemize}

\end{frame}

\begin{frame}{Reference 6}
\cite{tajalli2021traffic}
\begin{figure}
\includegraphics[height=5cm]{figs/ref6-1.png}
\end{figure}
\end{frame}

\begin{frame}{Reference 6}
\cite{tajalli2021traffic}

\begin{block}{}
\begin{itemize}
\item Objective function: maximize the traveled distance and minimize acceleration rates.
\item Collision avoidance: rear-end constraints and no running on red.
\item HDVs are predicted by Helly car-following model (modified to capture red signal), but same parameters for all HDVs.
\end{itemize}

\end{block}
\end{frame}


\begin{frame}{Challenges and Oppotunities}
\begin{block}{Notes}

\begin{itemize}
\item In all prior work, the traffic light for the lanes with conflicts can not be green simultaneously. Thus, the efficiency is limited if the CAV penetration rate is high.

\item How to predict human driving behavior given the traffic light? Most of prior work used well-known car-following models combined with heuristic rules given the traffic signal.

\item The optimization problem is large: 12 lanes with multiple vehicles on each lane.
The problem involves both continuous and discrete variables (mixed-integer optimization).
How to solve the problem but not in a centralized manner?

% \item Can approximate the traffic lights by continuous variables, \eg using sigmoid function, but it makes formulating the constraints hard (can use penalty function instead of hard constraint).  
\end{itemize}
	
\end{block}
\end{frame}

\begin{frame}{Scenario}
\begin{figure}
\centering
\includegraphics[width=0.7\textwidth]{figs/scenario_1.png}
% \caption{Caption}
% \label{fig:enter-label}
\end{figure}

\begin{block}{Notes}
\begin{itemize}
\item Turning-right lanes does not have possible lateral conflicts, thus can always be green. 12 lanes $\rightarrow$ 8 lanes.
\item Each lane has potential lateral conflict with 3 other lanes.
\end{itemize}
\end{block}
\end{frame}


\begin{frame}{Formulation \#1: Mixed-Integer Optimization}

Traffic Light System:

\begin{itemize}
\item Traffic light state: $l_l(k) \in \{0, 1\}$, $\forall l = 1, \dots, 12$ where $l_l(k) = 0$ if the traffic light is red and $l_l(k) = 1$ if the traffic light is green.
Note that yellow and red lights are combined and we consider a fixed cycle for yellow light. 


\item Let $t$ be the current time step and $k_{l,s} \in \RRplus$ (we treat $k_{l,s}$ as integer variables) be the time so that the traffic light switches at $t+k_s$, with $0 \le k_s \le H$. Then the traffic light system is modeled as follows:

If $l_l(t) = 0$,
\begin{equation*}
l_l (k) =
\begin{cases}
1, & \text{if } k \ge k_{l,s}, \\
0, & \text{otherwise,}
\end{cases}     
\end{equation*}
If $l_l(t) = 1$,
\begin{equation*}
l_l (k) =
\begin{cases}
0, & \text{if } k \ge k_{l,s} \\
1, & \text{otherwise}
\end{cases}     
\end{equation*}

\item Constraints for the minimum time between traffic light switches:
\begin{equation*}
\label{eq:min-swt}
k_{l,s} - k_0 \ge Q
\end{equation*}
where $k_0$ is the time of the last traffic light switch (known),
which means the gaps between two consecutive light switches must be more than $Q$ time steps.
\end{itemize}

\end{frame}

\begin{frame}{Formulation \#1: Mixed-Integer Optimization}
Some constraints:

\begin{itemize}

\item CAVs' speed and acceleration limit constraints:
\begin{equation*}
v_{\min} \le v_{l,i} (k+1) \le  v_{\max}, \; u_{\min} \le u_{l,i} (k+1) \le  u_{\max}    
\end{equation*}

\item No running on red light (does not apply to vehicles already crossed the stop line or not ready to stop):
\begin{equation*}
p_{i,l} (k) \le p_l^{\mathrm{stop}} \; \text{if} \; l_l(k) = 0.
\end{equation*}
that is equivalent to 
\begin{equation*}
p_{i,l} (k) \le p_l^{\mathrm{stop}} + M l_l(k).
\end{equation*}
where $M$ is a large positive constant.

\item Rear-end safety constraints for CAV if there is a preceding vehicle:
\begin{equation*}
p_{l,i} (k) + \rho v_{l,i} (k) + d_{min} \le p_{i-1,l} (k)
\end{equation*}

\end{itemize}
\end{frame}


\begin{frame}{Formulation \#1: Mixed-Integer Optimization}
Constraints (cont.):

\begin{itemize}
\item Lateral safety constraints between two CAVs if the lights for their lanes are both green (this constraint may also include binary variables)
\begin{equation*}
\begin{split}
p_{l,i} (k) - p_{l,m}^{\rm{conf}} & \ge d_{\rm{lat}} - M (c_{l,i,m,j} + e_{l,i,m,j}), \\
p_{l,m}^{\rm{conf}} - p_{l,i} (k) & \ge d_{\rm{lat}} - M (1 - c_{l,i,m,j} + e_{l,i,m,j}), \\
p_{m,j} (k) - p_{l,m}^{\rm{conf}} & \ge d_{\rm{lat}} - M (1 + c_{l,i,m,j} - e_{l,i,m,j}), \\
p_{l,m}^{\rm{conf}} - p_{m,j} (k) & \ge d_{\rm{lat}} - M (2 - c_{l,i,m,j} - e_{l,i,m,j}), 
\end{split}
\end{equation*}
where $c_{l,i,m,j}, e_{l,i,m,j} \in \{0, 1\}$.

\end{itemize}
\end{frame}

\begin{frame}{Formulation \#1: Mixed-Integer Optimization}
Constraints (cont.):

\begin{itemize}
\item Constraints to guarantee that there is no lateral conflicts between a CAV and an HDV or between two HDVs.
% Let lane-$k$ and lane-$l$ be two lanes with lateral conflicts, then 
\begin{equation*}
l_l(k) + l_m(k) \le 1 \;\text{if}\; b_{l,m} (k) > 0,
\end{equation*}
where $b_{l,m} > 0$ if there are a CAV and an HDV, or two HDVs traveling on lane-$k$ and lane-$l$ and having lateral conflicts.
Mathematically, it is
\begin{equation*}
\begin{multlined}
b_{l,m}(k) = \sum_{i=1}^{n_l} \sum_{j=1}^{n_m} (1 - c_{i,l} c_{j,m}) 
\Big[ 
\II \big( p_{i,l}(k) < p_{l,m}^{\text{conf}} \big) \II \big( p_{j,m}(k) < p_{m}^{\text{stop}} \big) \\
+ \II \big( p_{j,m}(k) < p_{l,m}^{\text{conf}} \big) \II \big( p_{i,l}(k) < p_{m}^{\text{stop}} \big)
\Big] 
\end{multlined}
\end{equation*}

This constraint is hard as it includes IF statement and the indicator function.
We can consider $b_{k,l} (k) = b_{k,l} (t)$, $\forall k \in \III_t$ as $b_{k,l} (k) \le b_{k,l} (t)$, $\forall k \in \III_t$, which results in more conservative constraints.
\end{itemize}
\end{frame}


\begin{frame}{Formulation \#1: Mixed-Integer Optimization}
\begin{itemize}
\item Objective function: Maximize the total progress along the path and minimize the acceleration rates of all vehicles, linearly combined as follows
\begin{equation*}
\sum_{l=1}^{12}  \sum_{k = t}^{t+H-1} - (n_{\text{veh},l}^{-}+1) \, l_l(k) + \sum_{i=1}^{n^C_l} \omega_v \big( v_{l,i} (k+1) - v_{\max} \big)^2 + \omega_u u^2_{l,i} (k),
\end{equation*}
where $n_{\text{veh}^{-},l}$ is the number of vehicles upstream the traffic light on lane-$k$.

\item The optimization problem is so far \textbf{mixed-integer quadratic programming (MIQP)}, but the objective is linear on the binary variables.

\end{itemize}




\end{frame}

\begin{frame}{Formulation \#1: Mixed-Integer Optimization}

\textbf{Summary:}

For TLC units:
\begin{itemize}
\item Local variables: $k_{l,s}$ (switching time), $l_l (k)$, $k \in \LLL_t$ (TL states), $c_{l,i,m,j}$, $e_{l,i,m,j}$ (binary variables for lateral constraints). \textcolor{red}{All the integer variables are handled by TLC units}

\item Objective: $\sum_{k = t}^{t+H-1} - (n_{\text{veh},l}^{-} + 1)\, l_l(k)$.

\item Local constraints: minimum time between traffic light switches.

\item Coupling constraints with TLC units of other lanes (no-conflict constraints between lanes).

\end{itemize}

For CAVs:
\begin{itemize}
\item Local variables: $p_{l,i}(k)$, $v_{l,i}(k)$, $u_{l,i}(k)$, $k \in \LLL_t$ (vehicle states and inputs). 

\item Objective: $\sum_{k = t}^{t+H-1} \omega_v \big( v_{l,i} (k+1) - v_{\max} \big)^2 + \omega_u u^2_{l,i} (k)$.

\item Local constraints: state and input bounds.

\item Coupling constraints: rear-end safety constraint, lateral safety constraint, no-running-on-red constraint.

\end{itemize}

\end{frame}

% \begin{frame}{Communication topology}

% \centering \includegraphics[width=0.8\textwidth]{figs/comm.png}

% \end{frame}



\begin{frame}{Formulation \#2: Nonlinear Optimization}

Using sigmoid function:
\[
\rm{sigmoid} (x) = \frac{1}{1+exp(-\mu x)}
\]

$l_l (k) \ge \frac{1}{2}$ indicates green light and $l_l (k) < \frac{1}{2}$ indicates red light. 


\centering \includegraphics[width=0.6\textwidth]{figs/sigmoid.pdf}
\end{frame}

\begin{frame}{Formulation \#2: Nonlinear Optimization}

\textbf{Note:} This approach only allows one traffic light switch over the control horizon.

\textbf{Fact:} The number of switches is limited due to the minimum time gap between switches.
Maximum number of switches is $S = \ceil{N/Q}$.

\begin{itemize}
\item Let $t$ be the current time step and $k_{l,s} \in \RRplus$ (we treat $k_{l,s}$ as real variables) be the time so that the traffic light switches at $t+k_{l,s}$, with $0 \le k_{l,s} \le H$. 
\item We approximate the traffic light state by sigmoid function of lane $l$ from $t$ to $t+H$ as follows
\begin{alignat*}{3}
l_l (k) &= \mathrm{sigmoid} (k-k_{l,s}) && \quad \text{if} \; l_l (t) = 0, \\
l_l (k) &= 1 - \mathrm{sigmoid} (k-k_{l,s}) && \quad \text{if} \; l_l (t) = 1,
\end{alignat*}
for $k \in \III_t$. 
\end{itemize}
\end{frame}




\begin{frame}{Formulation \#2: Nonlinear Optimization}
Constraints:
\begin{itemize}

\item Constraints to guarantee that there is no lateral conflicts between a CAV and an HDV or between two HDVs. The original constraint is 
\begin{equation*}
l_l (k) + l_m (k) \le 1, \; \forall k \in \III_t, \;\text{if}\; b_{l,m} (t) > 0 
\end{equation*}

If we use sigmoid function $0 < l_l, l_m < 1$, the constraint may be infeasible due to approximation errors.
Can use slack variable approach to handle infeasibility caused by the approximation errors.

\item Lateral collision avoidance constraint for two CAVs if the lights for their lanes are both green
\begin{equation*}
M(2-l_l(k)-l_m(k)) + \abs{p_{l,i}(k) - p_{l,m}^{\text{conf}}}^2 + \abs{p_{m,j}(k) - p_{l,m}^{\text{conf}}}^2 > (d^{\text{lat}})^2 
\end{equation*}

\item The other constraints and the objective function should be the same.
\end{itemize}
\end{frame}

% \begin{frame}{Next steps}
% \begin{block}{Notes on simulations}
% \begin{itemize}
% \item Both formulations work well in simulations, but the computation is not fast enough and not scalable to the number of CAVs. Thus, a distributed control framework is highly needed.
% \end{itemize}
% \end{block}

% \textbf{Challenge:
% \begin{itemize}
% \item How to solve the problem not in a centralized manner?
% \end{itemize}}

% \begin{figure}
% \centering
% \includegraphics[width=0.7\textwidth]{figs/structure.png}
% % \caption{Caption}
% % \label{fig:enter-label}
% \end{figure}
% \end{frame}


\begin{frame}{Distributed Optimization}
\begin{itemize}
\item Therefore, our problem (either mixed-integer or nonlinear ones) can be formulated as an optimization problem with separable objectives and pair-wise coupling constraints:
\begin{equation*}
\begin{split}
\underset{\bbsym{x}_i \in \XXX_i}{\minimize} & \quad \sum_{i=1}^{N} f_i (\bbsym{x}_i), \\
\text{subject to} & \quad h_{ij} (\bbsym{x}_i, \bbsym{x}_j) \le 0,
\end{split}
\end{equation*}
where $N$ is the total number of agents including CAVs and TLC units.

% \item The problem is equivalent to 
% \begin{equation*}
% \begin{split}
% \minimize & \quad \sum_{i=1}^{N} f_i (\bbsym{x}_i), \\
% \text{subject to} & \quad h_{ij} (\bbsym{x}_i, \bbsym{z}_{ij}) \le 0, \\
% & \quad \bbsym{x}_j - \bbsym{z}_{ij} = 0.
% \end{split}
% \end{equation*}
% where we introduce an auxiliary variable $\bbsym{z}_{ij}$ which is a copy of $\bbsym{x}_j$ and is stored by agent $i$.

\item \textbf{Note:} Research on the distributed solution of MIQP problems are relatively rare: [\cite{takapoui2020simple}], [\cite{liu2021distributed}]

\end{itemize}
\end{frame}

% \begin{frame}{Distributed Optimization}

% \begin{itemize}

% \item If CAVs need to keep the copy of binary variables in mixed-integer optimization, then we consider relaxation (\ie the integer variables take values in a convex hull of a discrete set).
% % As a result, the local optimization problem of each CAV will be a QP.


% \item The augmented Lagrangian is formulated as follows
% \begin{equation*}
% \LLL (\bbsym{x}_i, \bbsym{z}_{ij}) = \sum_{i=1}^{N} \left( f_i (\bbsym{x}_i) + \sum_{j \in \NNN_i^{+}} \II ( h_{ij} (\bbsym{x}_i, \bbsym{z}_{ij}) \le 0) 
% + \bbsym{\lambda}_{ij}^\top (\bbsym{x}_j - \bbsym{z}_{ij}) + \frac{\rho}{2} \norm{\bbsym{x}_j - \bbsym{z}_{ij}}_2^2 \right)
% \end{equation*}
% where $\bbsym{\lambda}_{ij}$ is the dual variable and $\rho > 0$ is a parameter.
% \end{itemize}

% \end{frame}

% \begin{frame}{Distributed Optimization}

% Then the ADMM algorithm works as follows
% \begin{enumerate}
% \item Agent $i$ updates the local variable
% \begin{equation}
% \begin{split}
% \bbsym{x}_{i}^{(k+1)}, \bbsym{z}_{i}^{(k+1)} &= \underset{\bbsym{x}_{i} \in \XXX_i, \bbsym{z}_{i} \in \ZZZ_i}{\argmin} \quad f_i(\bbsym{x}_{i}) + \frac{\rho}{2} \sum_{j \in \NNN_j^{+}} \norm{\bbsym{z}_{ij} - \bbsym{x}_{j}^{(k)} + \bbsym{\lambda}_{ij}^{(k)}/\rho}_2^2 \\
% & \text{subject to: } \quad h_{ij} (\bbsym{x}_i, \bbsym{z}_{ij}) \le 0
% \end{split}    
% \end{equation}

% \item Agent $i$ transmits $\bbsym{x}_{i}^{(k+1)}$ to agent $j \in \NNN_i^{-}$.

% \item Agent $i$ updates the dual variable
% \begin{equation}
% \bbsym{\lambda}_{ij}^{(k+1)} = \bbsym{\lambda}_{ij}^{(k)} + 
% \rho (\bbsym{x}_j^{(k+1)} - \bbsym{z}_{ij}^{(k+1)})
% \end{equation}
% \end{enumerate}

% \end{frame}


\begin{frame}{Distributed Optimization}

\begin{block}{Algorithm}
\begin{itemize}
\item Agent $i$ transmits the local variables $\bbsym{x}_i^{k}$ to agent $j \in \NNN_i$.

\item Agent $i$ updates the local variables given the other agents' variables.
\begin{equation*}
\begin{split}
\bbsym{x}_i^{(k+1)} = \quad
& \underset{\bbsym{x}_i \in \XXX_i}{\argmin} \quad f_i (\bbsym{x}_i), \\
& \text{subject to} \quad h_{ij} (\bbsym{x}_i, \bbsym{x}_j^{(k)}) \le 0,
\end{split}
\end{equation*}

\item Agent $i$ compute the local cost reduction $\Delta f_i^{(k+1)} = f_i(\bbsym{x}_i^{(k+1)}) - f_i(\bbsym{x}_i^{(k)})$.

\item Agent $i$ transmits the local cost reduction $\Delta f_i^{(k+1)}$ to agent $j \in \NNN_i$.

\item Agent $i$ accepts or rejects the new solution by comparing its local cost reduction with its neighbors. Only accept if its local cost reduction is the largest one among its neighbors. 

\end{itemize}
\end{block}

\textcolor{red}{It seems the algorithm will converge to a Nash equilibrium. Will prove it.}

\end{frame}

\begin{frame}{Next steps}

\begin{block}{}
\begin{itemize}
\item Implement the CAV lateral constraints (for now it is not included in the distributed algorithm). 

\item Prove the convergence to a Nash equilibrium.

\item Algorithms to improve the solutions.

\item Implement the optimization algorithm in C++ for higher computation speed (as parallelization in Python is bad).

\item Apply to the SUMO simulations with more lanes.
    
\end{itemize}
   
\end{block}

\end{frame}

% \begin{block}{Notes from simulations}

% \begin{itemize}
% \item The ADMM algorithm may converge to wrong solution for mixed-integer optimization. Relevance was observed in [\cite{takapoui2020simple}].

% \item Parallelization in Python is not good, need to write the algorithm in C++.
% \end{itemize}

% \end{block}

% \end{frame}


% \begin{frame}{Literature Review}

% \begin{itemize}
% \item Distributed optimization: \eg ADMM.
% \item Sequential computation with priority assignment:
% \end{itemize}

% \end{frame}


% \begin{frame}{HDV Prediction Model}
% \begin{figure}
% \centering
% \includegraphics[width=0.75\textwidth]{figs/singlelane.png}
% % \caption{Caption}
% % \label{fig:enter-label}
% \end{figure}

% \begin{block}{}
% The human driving behavior may follow different modes (\eg car-following, constant velocity, or constant acceleration), and the human control policy in each mode is parameterized by parameters (\eg car-following model's parameters).
% \end{block}

% \begin{block}{Definition}
% Vehicle index: Let a tuple $(i,k)$ be the index of the $i$-th vehicle traveling on lane $k$.
% \end{block}

% \end{frame}


% \begin{frame}{HDV Prediction Model}
% \begin{itemize}
% \item Let $\MMM$ be the set of operating modes, $M \in \MMM$ be the mode with associated uncertain parameters in a vector $\boldsymbol{\theta}^M$.
% The dynamics of the vehicles under each mode can be given by
% \begin{equation*}
% M \;:\; \bbsym{x}(k+1) = \bbsym{f}^M (\bbsym{x}(k), \bbsym{z}(k); \bbsym{\theta}^M) + \bbsym{w}^M (k),
% \end{equation*}
% where $\bbsym{z}(t)$ is the context variable, \eg states of preceding vehicles.

% \item We need to identify both the operating mode $M$ and the parameters $\boldsymbol{\theta}^M$.
% From Bayesian estimation, we compute $p \big( M, \bbsym{\theta}^M \,|\, \III(t) \big)$ where $\III(t)$ is the vector of available information at time $t$, \ie $\III(t+1)^\top = [\III(t)^\top, \bbsym{x} (t+1)^\top]$. 
% The posterior joint distribution of mode $M$ and $\bbsym{\theta}^M$ is given by
% \begin{equation*}
% p \big( M, \bbsym{\theta}^M \,|\, \III(t+1) \big)
% = p \big( \bbsym{\theta}^M \,|\, \III(t+1), M \big) 
% p \big( M \,|\, \III(t+1) \big).
% \end{equation*}

% \item Recursive Bayesian estimation:
% \begin{align*}
% p(\bbsym{\theta}^{M}\,|\, \III(k+1), M) &= \frac{p(\bbsym{x}(k+1) \,|\, \bbsym{z}(k), \III_k, M, \bbsym{\theta}^{M}) p(\bbsym{\theta}^{M}\,|\,  \III(k), M)}{p(\bbsym{x}(k+1) \,|\, \bbsym{z}(k), \III(k), M)}, \label{eq:Bayesian_update_param}
% \\
% p(M \,|\, \III(k+1)) & = \frac{p(\bbsym{x}(k+1) \,|\, \bbsym{z}(k), \III(k), M) p(M \,|\, \III(k))}{p(\bbsym{x}(k+1) \, | \, \bbsym{z}(k) ,\III(k))}.
% \end{align*}

% \end{itemize}

% \end{frame}

% \begin{frame}{HDV Prediction Model}
% If the system states are fully known, and the dynamics can be represented in the following form:
% \begin{equation*}
% M \;:\; \bbsym{x}(k+1) = \bbsym{f}_0^M (\bbsym{x}(k), \bbsym{z}(k)) + \bbsym{\Phi}^M (\bbsym{x}(k), \bbsym{z}(k))\,  \bbsym{\theta}^M + \bbsym{w}^M (k),
% \end{equation*}
% then we have a closed-form solution from \footcite{arcari2020dual}.

% For our problem: (1) car-following, (2) constant velocity, or (3) constant acceleration.
    
% \end{frame}

% \begin{frame}{HDV Prediction Model}

% \begin{itemize}
% \item Using CTH-RV car-following model:
% \begin{equation*}
% u_i(k) = \alpha (\Delta p_i(k) - \rho v_i(k)) + \beta \Delta v_t(k)
% \end{equation*}

% Thus
% \begin{equation*}
% \begin{split}
% p_i(k+1) &= p_i(k) + \tau v_i(k) + \frac{1}{2} \tau^2 u_i (k) \\
% v_i(k+1) &= v_i(k) + \tau u_i(k) \\
% u_i(k) & = \theta_1^{M_1} (p_{i-1}(k) - p_{i}(k) -d_{\min}) + \theta_2^{M_1} v_i(k) + \theta_3^{M_1} (v_{i-1}(k) - v_i(k))
% \end{split}
% \end{equation*}
% where $d_{\min}$ is a constant minimum inter-vehicle distance.

% Re-write in matrix form:
% \[\bbsym{x}_i (k) = [p_i (k), v_i (k)]^\top,\]
% \[\bbsym{z}_i (k) = \bbsym{x}_{i-1} (k),\]
% \[\bbsym{f}_0^{M_1} = [p_i (k) + \tau v_i(k), v_i (k)]^\top\]
% \[\bbsym{\Phi}^{M_1} = \begin{bmatrix}
% \frac{1}{2} \tau^2 (p_{i-1}(k) - p_{i}(k) -d_{\min}) & \frac{1}{2} \tau^2 v_i(k) & \frac{1}{2} \tau^2 (v_{i-1}(k) - v_i(k)) \\ 
% \tau (p_{i-1}(k) - p_{i}(k) -d_{\min}) & \tau v_i(k) & \tau (v_{i-1}(k) - v_i(k))
% \end{bmatrix}\]

% \end{itemize}
% \end{frame}

% \begin{frame}{HDV Prediction Model}
% \begin{itemize}

% \item Constant velocity:
% \begin{equation*}
% \begin{split}
% p_i(k+1) &= p_i(k) + \tau v_i(k) + \frac{1}{2} \tau^2 u_i (k) \\
% v_i(k+1) &= v_i(k) + \tau u_i(k) \\
% u_i(k) &= k_v (v^{\text{ref}} - v_i(k)) = \theta_1^{M_2} + \theta_2^{M_2} v_i(k)
% \end{split}
% \end{equation*}

% Re-write in matrix form:
% \[\bbsym{\Phi}^{M_2} = \begin{bmatrix}
% \frac{1}{2} \tau^2 & \frac{1}{2} \tau^2 v_i(k) \\ 
% \tau & \tau v_i(k) 
% \end{bmatrix}\]

% \item Constant acceleration:
% \begin{equation*}
% \begin{split}
% p_i(k+1) &= p_i(k) + \tau v_i(k) + \frac{1}{2} \tau^2 u_i (k) \\
% v_i(k+1) &= v_i(k) + \tau u_i(k) \\
% u_i(k) &= a^{\text{ref}} = \theta_1^{M_3}
% \end{split}
% \end{equation*}

% Re-write in matrix form:
% \[\bbsym{\Phi}^{M_3} = \begin{bmatrix}
% \frac{1}{2} \tau^2 \\ 
% \tau 
% \end{bmatrix}\]
% \end{itemize}
% \end{frame}

\end{document}
